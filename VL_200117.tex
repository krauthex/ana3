\documentclass[skript.tex]

\begin{document}
	%\setcounter{cntr}{25}
	%\setcounter{section}{3}
	\section{Glatte Ränder}
	
	\begin{defin}[Relativtopologie und Rand]
		Sei $(X,\mc{O})$ ein topologischer Raum, $Y \sbs X$ eine nichtleere Teilmenge von $X$. Wir bezeichnen
		\[
			\mc{O} \cap Y \ceq \cbr{U \cap Y \mid U \in \mc{O}}
		\]
		als \textit{Relativtopologie} von $Y$ bezüglich $X$.
		
		Der Rand $\del Y$ ist die Menge aller Punkte $x\in X$, für die jedes $U \in \mc{O}$	mit $x\in U$ Punkte aus $Y$ und $\cp{Y}=X\sm Y$ enthält.Insbesondere ist $(Y, \mc{O}\cap Y)$ wieder ein topologischer Raum.
		\missingGraphic
	\end{defin}
	
	
	\begin{defin}[Glatte Ränder und adaptierte Karten]
		Sei $M\sbs \Rn$ eine m-dimensionale Mannigfaltigkeit und $\Omega \sbs M$. Wir sagen $\Omega$ habe einen \textit{glatten Rand}, falls es für jedes $p\in \del \Omega$ eine Karte $\varphi^{-1} \colon W \lra U$ mit $p\in W$ und $\varphi\pr{U \cap \cbr{x_1 \leq 0}} = \Omega \cap W$ sowie \\$\varphi\pr{U \cap \cbr{x_1 = 0}} = \del \Omega \cap W$ gibt. Eine solche Karte $\varphi^{-1}$ heißt \textit{$\Omega$-adaptiert.} Ein Atlas heißt \textit{$\Omega$-adaptiert}, falls sämtliche seiner Karten deren Definitionsbereich $\del \Omega$ schneidet, $\Omega$-adaptiert sind.
		\missingGraphic
	\end{defin}
	
	\begin{lem}
		Sei $M$ eine $m$-dimensionale Mannigfaltigkeit im $\Rn, \Omega \sbs M$ eine Teilmenge mit glattem Rand. Dann gibt es einen $\Omega$-adaptierten Atlas. Ist $M$ orientiert und $m\geq 2$, so kann man erreichen, dass dieser Atlas orientiert ist.
	\end{lem}
	
	\begin{proof}
		Nach Definition erhalten wir eine (abzählbare) Familie in $M$ offener Mengen, die $\del \Omega$ überdecken und auf denen jeweils Karten definiert sind. Der Definitionsbereich aller weiteren Karten des Atlas schränken wir so ein, dass kein Punkt aus $\del \Omega$ enthalten ist.
		
		Durch Ausnutzen der Glattheit des Randes erhalten wir so aus einem gegebenen Atlas $\mc{A}_0$ einen $\Omega$-adaptierten Atlas $\mc{A}$. Wir nehmen nun an, dass $\mc{A}_0$ orientiert ist. Indem wir bei der Konstruktion von $\mc{A}$ die Definitionsbereiche entsprechend einschränken, finden wir für jede Karte $\varphi^{-1}\colon W\lra U$ aus $\mc{A}$ ein $\wt{\varphi}^{-1} \colon \wt{W} \lra \wt{U}$ aus $\mc{A}_0$ mit $W\sbs \wt{W}$, wobei $\det D\pr{\wt{\varphi}^{-1} \circ \varphi}$ auf $W$ das Vorzeichen nicht ändert. Ersetzen wir nun sämtliche Karten $\varphi^{-1} \in \A$, für die dieses Vorzeichen negativ ist, durch $\pr{\varphi \circ S}^{-1}$, wobei $S$ die Spiegelung an $\pr{\R \times m}^\bot$ ist, das heißt, $S e_j = e_j$ für $j=1,\dots, m-1,\, S e_m = -e_m$, so erhalten wir einen $\Omega$-adaptierten Atlas $\wh{\A}$ mit $\det \pr{\wt{\varphi}^{-1} \circ \varphi} > 0\ \forall \varphi^{-1} \in \A$ (und $\wt{\varphi}$ wie eben gefordert). Für Karten \\ $\varphi_1^{-1} \colon W_1 \lra U_1, \, \varphi_2^{-1} \colon W_2 \lra U_2$ aus $\wh{\A}$ und $W_1 \cap W_2 \neq \es$ und $\wt{\varphi}_1^{-1}\colon \wt{W}_1 \lra \wt{U}_1,\, \\ \wt{\varphi}_2^{-1}\colon \wt{W}_2 \lra \wt{U}_2$, wobei $W_1 \sbs \wt{W}_1$ und $W_2 \sbs \wt{W}_2$, gilt nun:
		\[
			\varphi_2^{-1} \circ \varphi_1 = \underbrace{\pr{\varphi_2^{-1} \circ \wt{\varphi}_2}}_{=\pr{\wt{\varphi}_2^{-1} \circ \varphi_2} } \circ \pr{\wt{\varphi}_2^{-1} \circ \wt{\varphi}_1} \circ \pr{\wt{\varphi}_1^{-1} \circ \varphi_1},
		\]
		und dies ist orientierungstreu, da Verkettungen und Umkehrungen orientierungstreuer Abbildungen wieder orientierungstreu sind. Die Karten aus $\A_0$ und $\wh{\A}$ sind gleichorientiert, denn fur eine Karte $\wt{\varphi}_1^{-1} \colon \wt{W}_1 \lra \wt{U}_1$ aus $\A_0$ und eine weitere Karte $\varphi_2^{-1} \colon W_2 \lra U_2$ aus $\wh{A}$ mit $W_1 \cap W_2 \neq \es$ ist der Kartenwechsel 
		\[ 
			\varphi_2^{-1} \circ \varphi_1 = \underbrace{\pr{\varphi_2^{-1} \circ \wt{\varphi}_1}}_{\A_0\text{-Kartenwechsel}} \circ \underbrace{\pr{\wt{\varphi}_1^{-1} \circ \varphi_1}}_{\mathrlap{\text{orientierungstreu nach Karten von } \wh{\A}}}
		\]
	auf seinem Definitionsbereich orientierungstreu.
	\end{proof}
	
	\missingGraphic
	
	\begin{theorem}[Ränder als Mannigfaltigkeiten]
		Sei $M$ eine m-dimensionale Mannigfaltigkeit im $\Rn,\\ m,n \in \N,\, m\leq n$. Ist $\Omega \sbs M$ eine Teilmenge mit glattem Rand, so ist $\del M$ eine $(m-1)$-dimensionale Mannigfaltigkeit im $\Rn$. Ist $M$ orientierbar, so ist auch $\del M$.
	\end{theorem}

	\begin{proof}
		Für einen $\Omega$-adaptierten Atlas $\A$ und $\varphi^{-1} \colon W\lra U$ aus $\A$ mit $\del \Omega \cap W \neq \es$ definieren wir eine stetige, bijektive Abbildung $\wt{\varphi}^{-1} \colon \del \Omega \cap W \lra \wt{U}$, wobei 
		\[
			\wt{U} \ceq \cbr{x' \in \R^{m-1} \mid \pr{0, x'} \in U \sbs \R^m},\quad \wt{\varphi}(x') = \varphi(0,x'),\, x' \in \wt{U}.
		\]
		Wenn $U$ offen ist, so ist auch $\wt{U}$ eine offene Teilmenge in $\R^{m-1}$. Weiterhin ist $\del \Omega \cap W$ offen in $\del \Omega$. Weil auch die Umkehrabbildung $\wt{\varphi}^{-1} = P\circ \varphi^{-1}\vert_{\del \Omega \cap W}$ stetig ist, wobei $P$ die Projektion $(x_1, x') \mapsto x'$ bezeichnet, ist $\wt{\varphi}$ ein Homöomorphismus. Wegen $D\wt{\varphi} = \pr{\ddel{\varphi}{x_2}, \dots, \ddel{\varphi}{x_m}}$ und $\rang D\varphi =m$ hat $D\wt{\varphi}$ vollen Rang mit $m-1$ und $\del M$ ist eine Mannigfaltigkeit. Nun sei M orientiert. Fall $m=1$ führt auf Punktmengen. Also $m\geq 2$. Nach \textit{Lemma 28} können wir annehmen, dass $\A$ aus gleichorientierten $\Omega$-adaptierten Karten besteht. Zu zeigen bleibt, dass nun auch alle Karten aus dem Atlas $\wt{\A}$ gleich orientiert sind, wobei $\wt{\A}$ der Atlas für $\del \Omega$ ist. der aus offenen Karten $\wt{\varphi}^{-1} $ besteht, die wie oben aus adaptierten Karten $\varphi^{-1} \in \A$ hervorgehen, deren Definitionsbereich $\del \Omega$ schneidet. Hierzu betrachten wir Karten\\ $\varphi_1^{-1} \colon W_1 \lra U_1,\ \varphi_2^{-1} \colon W_2 \lra U_2$ aus $\A_0$ mit $W_1 \cap W_2 \cap \del\Omega \neq \es$. Wie oben erhalten wir hieraus Karten $\wt{\varphi}_1^{-1} \colon \del\Omega \cap W_1 \lra \wt{U}_1,\ \wt{\varphi}_2^{-1} \colon \del\Omega \cap W_2 \lra \wt{U}_2$ aus $\wt{\A}$. Der Kartenwechsel $\psi \ceq \wt{\varphi}_2^{-1} \circ \wt{\varphi}_1 \colon \varphi_1^{-1}\pr{\del\Omega \cap W_1 \cap W_2} \lra \varphi_2^{-1} \pr{\del\Omega \cap W_1 \cap W_2}$ ist durch \\$\wt{\psi}(x') = P\circ \varphi_2^{-1} \circ \varphi_1(0, x') = \pr{\psi_2(0,x'),\dots,\psi_m(0,x')}$, gegeben, wobei \\$(0,x') \in \varphi_1^{-1} \pr{\del\Omega \cap W_1 \cap W_2}$ und $\psi = \varphi_2^{-1} \circ \varphi_1 \colon \varphi_1^{-1} (W_1 \cap W_2) \lra \varphi_2^{-1} (W_1 \cap W_2)$ ein Kartenwechsel ist. Man erhält $D\wt{\psi}(x')$ als rechte untere $(m-1)\times(m-1)$-Untermatrix von $D\psi(0,x')$.
		
		Weil nun $\varphi_1^{-1}, \varphi_2^{-1}$ beide $\Omega$-adaptiert sind, gilt:
		\[
			\psi_1 (0,x_2, \dots, x_m) = 0 \text{ und } \psi_j(x_1, x_2, \dots,x_m) \leq 0 \text{ für } x_1 \leq 0.
		\]
		Damit ist 
		\[
			\ddel{\psi_1}{x_j} (0,x_2, \dots, x_m) 
				\begin{cases}
					\geq 0 &\text{für } j=1\\
					=0 &\text{für } j=2,\dots,m
				\end{cases}\ .
		\]
		Insofern haben wir 
		\[
			0 < \det D\psi(0,x') = \det 
				\begin{pmatrix}
					\ddel{\psi_1}{x_1} & \text{\raisebox{-0.4em}{\huge0}} \\ 
					&\\
					\text{\raisebox{-0.7em}{\huge*}} & D\wt{\psi}(x')
				\end{pmatrix} = \underbrace{\ddel{\psi_1}{x_1}}_{>0} \det D\wt{\psi}(x'),
		\]
		also $\det D\wt{\psi}(x') > 0$.
	\end{proof}
	
\end{document}