\documentclass[skript.tex]{subfiles}

\begin{document}
	\chapter{Lebesgue-Integral}
	\section{Maßräume}
	
	\begin{defin}[Algebra]
		Eine Algebra $\mc{A}$ ist eine Familie von Teilmengen einer gegebenen Menge $X$ mit folgenden Eigenschaften:
		\begin{itemize}
			\item $X \in \mc{A}$
			\item $A \in \mc{A} \Rightarrow \cp{A}\coloneqq X\setminus A \in \mc{A}$
			\item $A_1,...,A_N \in \mc{A}, N \in \mathbb{N} \Rightarrow \bigcup_{k=1}^N A_k \in \mc{A}$
		\end{itemize}
	Wir sprechen von einer $\sigma$-Algebra, wenn $N=\infty$ zulässig ist.
	\end{defin}

	\begin{lem}
		Sei $X$ eine Menge. $\mc{A}$ eine $\sigma$-Algebra, $\pr{A_k}_{k\in\mathbb{N}} \subset \mc{A}$. Dann gehören auch $\bigcap_{k=1}^\infty A_k$ und beispielsweise $A_1 \setminus A_2$ zu $\mc{A}$. 
	\end{lem}
	\begin{proof}
		Wir haben
			\begin{equation*}
				\bigcap_{k\in\mathbb{N}} A_k = \underbrace{\pr{\bigcup_{k\in\mathbb{N}} \underbrace{\cp{A_k}}_{\in\ \mc{A} } }}_{\in\ \mc{A}}  \raisebox{1.7em}{\kern -0.1em C}  \in \mc{A}
			\end{equation*}
		Weiter ist $A_1\setminus A_2 = A_1 \cap \underbrace{\cp{A_2}}_{\in\ \mc{A}} \in \mc{A}$
	\end{proof}
	
	\begin{bsp*}
		Für $X=\left\{ 1,2,3\right\}$ ist $\mc{A}=\left\{\emptyset, X, \left\{1\right\}, \left\{ 2,3\right\} \right\}$
	\end{bsp*}
	
	\begin{defin}
		Allgemein ist $\mf{P}\pr{X}$ (=Potenzmenge, Menge aller Teilmengen von X) die größte und $\left\{\emptyset, X \right\}$ die kleinste $\sigma$-Algebra. Sei $S \subset \mf{P}\pr{X}$, dann stellt
		\begin{equation*}
			\Sigma\pr{S} = \bigcap\left\{ \mc{A} \mid \mc{A}\ \sigma\text{-Algebra mit } S \subseteq \mc{A}\right\}
		\end{equation*}
		tatsächlich eine $\sigma$-Algebra dar. Es ist die kleinste $\sigma$-Algebra die $S$ enthält und wird als \textit{die von $S$ erzeugte} $\sigma$-Algebra bezeichnet. $\Sigma\pr{S}$ ist eindeutig bestimmt.
		
		Ist $X$ eine Menge mit $\sigma$-Algebra $\mc{A}$ und $Y \subset X$. Dann bezeichnen wir
		\begin{equation*}
			\mc{A}\cap Y := \cbr{A\cap Y \mid A \in \mc{A} }
		\end{equation*}
		als \textit{relative $\sigma$-Algebra}. Sie ist in der Tat eine $\sigma$-Algebra auf Y.
	\end{defin}

	\begin{lem}
		Die erzeugte und die relative $\sigma$-Algebra sind wohldefiniert, also eindeutig bestimmt, und tatsächlich $\sigma$-Algebren.
	\end{lem}
	\begin{proof}
		In der Übung.
	\end{proof}

	\begin{beh*}
		Falls $S\subset \mf{P}\pr{X}$ die $\sigma$-Algebra $\Sigma = \Sigma\pr{S\mid X}$\footnote{Hier bezeichnet $X$ den ''Raum'', der für $\Sigma$ von Bedeutung ist. } erzeugt, dann erzeugt für $Y \subset X$ die Menge $S \cap Y$ die $\sigma$-Algebra $\Sigma\pr{S\cap Y\mid Y}$, und
		\begin{equation*}
			\Sigma\pr{S\cap Y \mid Y} = \Sigma\pr{S\mid X} \cap Y
		\end{equation*}
	\end{beh*}
	\begin{proof}\hfill 
		\begin{itemize}
			\item[''$\subseteq$''] Weil $\Sigma \cap Y$ die Mengen aus $S \cap Y$ enthält, gilt $\Sigma\pr{S\cap Y} \subseteq \Sigma\pr{\Sigma \cap Y} = \Sigma \cap Y$ 
			\item[''$\supseteq$''] Betrachte die Menge
				\begin{equation*}
					\cbr{A \subset \Sigma\pr{S} \mid A \cap Y \in \Sigma\pr{S\cap Y}}
				\end{equation*}
				Dies ist eine $\sigma$-Algebra. Weil diese die Menge $S$ enthält, folgt
				\begin{equation*}
					\Sigma\pr{S} \subset \cbr{A \in \Sigma\pr{S} \mid A \cap Y \in \Sigma\pr{S\cap Y}} \subset \Sigma\pr{S}
				\end{equation*}
				also Gleichheit und folglich 
				\[
					\Sigma\pr{S} \cap Y = \cbr{A \in \Sigma\pr{S} \mid A \cap Y \in \Sigma\pr{S \cap Y}} \cap Y \subset \Sigma\pr{S \cap Y}
				\]
		\end{itemize}
	\end{proof}
	
	\begin{defin}[Topologischer Raum]
		Sei $X$ eine Menge. Es gibt ein System von Teilmengen $\mc{O} \subset \mf{P}(X)$ mit $\emptyset,X \in \mc{O}$, das abgeschlossen ist unter endlichen Schnitten und abzählbaren Vereinigungen. Dieses System $\pr{X,\mc{O}}$ heißt \textit{topologischer Raum}. Formal muss gelten:
		\begin{itemize}
			\item $\emptyset, X \in \mc{O}$
			\item $U,V \in \mc{O} \Rightarrow U\cap V \in \mc{O}$
			\item $\cbr{U_i}_{i\in I}$ , $U_i \in \mc{O} \Rightarrow \bigcup_{i\in I} U_i \in \mc{O}$
		\end{itemize}
	\end{defin}
	
	\begin{defin}[Borel-$\sigma$-Algebra]
		Ist $X$ ein topologischer Raum, $\mc{O}\subset \mf{P}(X)$, so ist $\mc{B}\pr{X}$ diejenige $\sigma$-Algebra, die von $\mc{O}$ erzeugt wird (also die kleinste $\sigma$-Algebra, die die offenen Mengen von $\pr{X,\mc{O}}$ enthält). Wir bezeichnen $\mc{B}\pr{X}$ als \textit{Borel-$\sigma$-Algebra}, und die Mengen in $\mc{B}$ heißen \textit{Borel-Mengen}.
		
		Notation:
		\begin{equation*}
			\mc{B}^n = \mc{B}\pr{\mathbb{R}^n},\ \mc{B} = \mc{B}^1
		\end{equation*}
	\end{defin}

	\begin{bem*}
		Die Familie aller endlichen offenen Intervalle $\subset \mathbb{R}$ erzeugt bereits $\mc{B}$.
	\end{bem*}
	
	\begin{defin}[Messraum, Maß, Maßraum]
		Eine Menge $X$ mit einer $\sigma$-Algebra $\mc{A} \subset \mf{P}\pr{X}$ heißt \textit{Messraum}. Ein \textit{Maß} ist eine Abbildung $\mu\colon \mc{A} \longrightarrow \left[ 0, \infty \right]$ mit:
		\begin{itemize}
			\item $\mu\pr{\emptyset} = 0$
			\item \textbf{\boldmath$\sigma$-Additivität}: Für eine Folge\footnote{Folgen sind indizierbar mit $\mathbb{N}$. Im Unterschied dazu können Familien auch überabzählbar sein.} paarweise disjunkter\footnote{$A_j \cap A_k = \emptyset$ für $j\neq k$} Mengen $\pr{A_k}_{k\in \mathbb{N}} \subset \mc{A}$ ist \\ $\mu\pr{\bigcup_{k\in\mathbb{N}} A_k } = \sum_{k\in\mathbb{N}} \mu\pr{A_k}$
		\end{itemize}
		Die Elemente in $\mc{A}$ heißen \textit{messbar}, und das Tripel $\pr{X, \mc{A}, \mu}$ heißt somit \textit{Maßraum}.
	\end{defin}
	
	\begin{defin}[$\sigma$-Finitheit]
		Ein Maß heißt \textit{$\sigma$-finit}, falls es eine abzählbare Überdeckung \\ $\cbr{X_k}_{k\in\mathbb{N}} \subset \mc{A}$ von $X$ gibt, also $X=\bigcup_{k\in\mathbb{N}} X_k$, sodass $\mu\pr{X_k} < \infty\ \forall k$.
		
		$\mu$ heißt \textit{endlich}, falls $\mu\pr{X}<\infty$, und \textit{Wahrscheinlichkeitsmaß}, falls $\mu\pr{X}=1$.
	\end{defin}

	\begin{bsp}
		\hfill
		\begin{itemize}
			\item[(a)] \textbf{Zählmaß}: Für $X$ und $\mc{A}=\mf{P}\pr{X}$ setze für $A \in \mc{A}$
				\begin{equation*}
					\mu\pr{A} = 
					\begin{cases}
						\#A &: A \text{ endlich}\\
						\infty &: \text{sonst}
					\end{cases}
				\end{equation*}
				$\mu$ ist endlich, wenn $X$ endlich, und $\sigma$-finit, wenn $X$ abzählbar ist.
				
			\item[(b)] \textbf{Dirac-Maß}: Für einen fest gewählten Punkt $x_0 \in X$ und $\mc{A}=\mf{P}\pr{X}$ setze für $A \subset X$
				\begin{equation*}
					\mu\pr{A} = 
						\begin{cases}
							0 &: x_0 \notin A\\
							1 &: x_0 \in A
						\end{cases}
				\end{equation*}
			
			\item[(c)] \textbf{Positive Linearkombinationen}: Seien $\mu_1, \mu_2$ Maße auf $\pr{X, \mc{A}}$. Dann erhalten wir durch $\mu:= \alpha_1\mu_1 + \alpha_2\mu_2$ für $\alpha_1, \alpha_2 \geq 0$ wieder ein Maß. 
		\end{itemize} 
	\end{bsp}

	\begin{bsp}
		Sei $\mu$ ein Maß auf $\pr{X,\mc{A}}$ und $Y\in\mc{A}$. Dann ist $\mu|_{Y}\pr{A} \coloneqq \mu\pr{A\cap Y}$ wieder ein Maß auf $\pr{X, \mc{A}}$. 
	\end{bsp}
	
	\begin{bem*}
		Für $Y \in \mc{A}$ können wir die $\sigma$-Algebra $\mc{A}$ zu 
		\begin{equation*}
			\mc{A}|_{Y} = \cbr{A \in \mc{A} \mid A \subset Y}
		\end{equation*}
		einschränken. Dann ist $\mu|_Y\pr{A} = \mu\pr{A\cap Y},\ A\cap Y \in \mc{A},$ ein Maß (siehe oben) und $\pr{Y,\mc{A}|_Y,\mu|_Y}$ ein Maßraum und dieser ist $\mu$-finit, falls $\pr{X,\mc{A},\mu}\ \sigma$-finit ist. 
	\end{bem*}
	
	\begin{notat}
		\hfill
		\begin{itemize}
			\item[] $A_k \kern -1pt \nearrow \kern -2pt  A$, falls $A_k \subset A_{k+1}\ \forall k\in \mathbb{N}$ und $A = \bigcup_{k\in\mathbb{N}} A_k$
			\item[] $A_k \kern -2pt \searrow \kern -1pt  A$, falls $A_k \supset A_{k+1}\ \forall k\in \mathbb{N}$ und $A = \bigcap_{k\in\mathbb{N}} A_k$
		\end{itemize}
	\end{notat}

	\begin{theorem}
		Für jeden Maßraum $\pr{X,\mc{A},\mu}$ und $\pr{A_k}_{k\in\mathbb{N}} \subset \mc{A}$ gilt:
		\begin{itemize}
			\item[(i)] $A \subset B \Rightarrow \mu\pr{A} \leq \mu\pr{B}$ \textup{(Monotonie)}
			\item[(ii)] $\mu\pr{\bigcup_{k\in\mathbb{N}}} \leq \sum_{k\in\mathbb{N}} \mu\pr{A_k}$ \textup{($\sigma$-Subadditivität)}
			\item[(iii)] $A_k\nearr A\Rightarrow \mu\pr{A_k}\nearr \mu\pr{A}$
			\item[(iv)] $A_k\searr A\Rightarrow \mu\pr{A_k}\searr \mu\pr{A}$\textit{, für }$\mu\pr{A_1} < \infty$
		\end{itemize}
	\end{theorem}
\end{document}