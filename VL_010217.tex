\documentclass[skript.tex]{subfiles}


\begin{document}
	
	\begin{theorem}
		Für offene Mengen $\Omega_1 \sbs \Rn, \: f \in C^2 (\Omega_1,\Omega_2)$ und eine differenzierbare Differenzialform auf $\Omega_2$ ist auch die auf $\Omega_1$ zurückgeholte Form $f^*\omega$ differenzierbar und es gilt $\md( f^* \omega) = f^*(\md \omega)$.
	\end{theorem}

	\begin{proof}
		Für eine differenzierbare Differentialform der Ordnung $O$ ist $f^*g = g \circ f$ differenzierbar und für $x \in \Omega_1$ und $v \in \R^m$ haben wir
		
		\begin{equation*}
		\md(f^*g)(v)= \md(g \circ f)(v)= \md g (f(x)) \md f(x)v = f^*(\md \omega)(x)(v)
		\end{equation*}
		
		Für allgemein $\omega = \sum_{j_1 < \dots j_k} a_{j_1, \dots,j_k} \md x_{j_1} \wedge \dots \wedge \md x_{j_k}$ erhalten wir mit Lemma 9
		
		\begin{align*}
		f^*\md \omega &= \sum_{j_1 < \dots < j_k} \underbrace{f^* \md a_{j_1, \dots, j_k}}_{=\md (f^*a_{j_1, \dots, j_k})} \wedge \underbrace{f \md x_{j_1}}_{= \md (f^ x_{j_1})} \wedge \dots \wedge \underbrace{f^* \md x_{j_k}}_{= \md (f^* x_{j_k})}\\
		 &= \md ( \sum _{\mathclap{j_1 < \dots < j_k}} (f^*a_{j_1, \dots j_k}) \wedge \md(f^* x_{j_1}) \wedge \dots \wedge \md (f^* x_{j_k}) = \md( f^* \omega).
		\end{align*}
		
	\end{proof}
		
		\section{Integration von Differentialformen}
		\begin{defin}
			Sei $\Omega \sbs \Rn$ offen. Eine Differentialform $\omega = f \md x_1 \wedge \dots \wedge \md x_n$ heißt integrierbar über $A \sbs \Omega$ falls $f$ über $A$ integrierbar ist. Wir setzten 
			
			\begin{equation*}
			\int_A \omega = \int_A f \md \lambda^n.
			\end{equation*}
		\end{defin}
	
	\begin{theorem}[Transformationsformel]
		Sind $U,V \sbs \Rn$ offen, $\varphi \colon V \to U$ ein orientierungstreuer $C^1$- Diffeomorphismus und $\omega$ eine integrierbare Differentialform der Ordnung $n$ auf $U$ so gilt
		
		\begin{equation*}
		\int_V \varphi^* \omega = \int_U \omega.
		\end{equation*}
		
		Im allgemeinen Fall $k \in \{1, \dots, n\}$ definieren wir Integrale zunächst über lokale Parametrisierung.
		
		\begin{proof}
			Für $\omega = f \md x_1 \wedge \dots \wedge \md x_n$ erhalten wir zunächst 
			\begin{align*}
			(\varphi^* \omega)(x)(v_1, \dots, v_2) &= \omega(\varphi(x))(\md \varphi(x) v_1, \dots, \varphi(x)v_n) \\
			&= f(\varphi(x))(\md x_1 \wedge \dots \wedge \md x_n)(\md \varphi(x)v_1, \dots, \md \varphi(x) v_n)\\
			&= f(\varphi(x))\underbrace{(\md \varphi(x))^*}_{\det \md \varphi(x)}(\md x_1 \wedge \dots \wedge \md x_n)(v_1, \dots, v_n),
			\end{align*}
			
			also
			
			 \begin{equation*}
			 \int_V \varphi^* \omega = \int_V f \varphi \underbrace{\det \md \varphi( \cdot)}_{>0} \md \lambda^n \overset{I.70}{=} \int_U f \md \lambda^n = \int_U \omega.
			 \end{equation*}
			
			
		\end{proof}
	\end{theorem}
	
		\begin{defin}
			Sei $\Omega \sbs \Rn$ und $M \sbs \Omega$ eine k-dimensionale orientierte Mannigfaltigkeit sowie $\varphi^{-1} \colon W \to U$ eine Karte eines orientierten Atlanten. Eine auf $ M\sm W$ verschwindende Differentialform $\varphi^* \omega$ auf $U$ im Sinne von Definition 5.19 integrierbar ist, und wir setzten
		
			\begin{equation*}
			\int_M \omega = \int_U \varphi^* \omega
			\end{equation*}
			
			Wie bei der Definition von Flächenintegralen muss man sich von der Unabhängigkeit dieser Definition von der gewählten Karte überzeugen; dies folgt aus der angegebenen Transformationsformel. Eintsprechendes gilt für die folgende Konvention.
		\end{defin}
	
		\begin{defin}
			Sei $ \Omega \sbs \Rn$ offen, $M \sbs \Omega$ eine k-dimensionale Mannigfaltigkeit mit orientiertem Atlas $(\varphi_j^{-1} \colon W_j \to U_j)_{j = 1, \dots, N}$. Eine Differentialform $\omega \colon \Omega \to \alt^k \Rn$ heißt integrierbar, falls $\rchi_{W_j} \omega$ für alle 
			$j = 1, \dots , N$ im Sinne von Definition 5.21 integrierbar ist. Ist $(\alpha_j)_{j=1, \dots, N}$ eine der Überdeckung $(W_j)_{j = 1, \dots, N}$ untergeordnete Partiion der Eins und $\alpha_j \circ \varphi_j$ messbar für $j = 1, \dots , N$, so definieren wir das Integral von $\omega$ über $M$ durch 
			
			\begin{equation*}
			\int_M \omega = \sum_{j=1}^{N} \int_M \alpha_j \omega
			\end{equation*}
			
			wobei auf der rechten Seite die in Definition V.21 erklärten Integrale stehen.
		\end{defin}
	
		\begin{bsp}[Kurvenintegrale]
			Sei $\gamma \in C^1(I, \Rn)$ für endliches Intervall $I = (a,b), \: 0<c \leq \abs{\gamma '} \leq C < \infty$ auf $I$ und $ \gamma \colon I \to \image\gamma$ ein Homöomorphismus, so ist das $\image\gamma$ eine eindimensionale Mannigfaltigkeit, und für eine Differentialform $\nu = \sum_{j=1}^{n} f_j \md x_j$erhalten wir die zurückgeholte Form $\gamma^* \nu = \sum_{j=1}^{n} (f_j \circ \gamma) \md \gamma_j$. Ist $\nu$ integrierbar, so folgt
			
			\begin{equation*}
			\int_{\image \gamma} \nu = \int_I \gamma^* \nu = \int_I \skp{f(\gamma(t))}{\gamma'(t)} \md t.
			\end{equation*}
			
			Sei nun $\omega$ eine stetig differenzierbare Funktion auf einer Umgebung von Bild $\gamma$. Dann haben wir für $\nu = \md \omega= \sum_{j=1}^{n} \frac{\del \omega}{\del x_j} \md x_j$
			
			\begin{equation*}
			\int_{\image\gamma} \md \omega = \int_I \skp{\nabla \omega( \gamma(t))}{\gamma' (t)} \md t = \omega(\gamma(b)) - \omega(\gamma(a)) \quad\text{„}= \int_{\del \image \gamma} \omega\text{“}
			\end{equation*}
			
			\end{bsp}
			
			\begin{lem}
				Für eine stetig differenzierbare Differentialform $\omega$ der Ordnung $k-1$ mit kompaktem Träger auf $\R^k, \: k\geq 2$, ist
				
				\begin{equation*}
				\int_{\{x_1 \leq 0\}} \md \omega = \int_{ \del \{x_1 \geq 0\}} \omega
				\end{equation*}
				
				\begin{proof}
					Für $\omega = \sum_{j =1}^{k}(-1)^{j-1} f_j \md x_1 \wedge \dots \wedge \md x_{j-1} \wedge \md x_{j+1} \wedge \dots \wedge \md x_k $ mit $f_j \in C^1 (\R^k)$ folgt
					
					\begin{equation*}
					 \int_{\{x1 \leq 0\}} \md \omega \overset{\text{Bsp} 17}{=} \int_{\{x_1 \leq 0\}}(\text{div} f) \md x_1 \wedge \dots \wedge \md x_k \overset{\text{Def} 19}{=}  \int_{\{x_1 \leq 0\}} \text{div} f \md \lambda^n
					 \end{equation*}
					
					Mit dem Satz von Fubini können wir den j-ten Summanden in der Divergenz $\text{div} f = \sum_{j=1}^{n} \frac{\del f}{\del x_j}$ zunächst und $x_j$ integrieren. Wir erhalten
					
					\begin{align*}
					&\int_{-\infty}^{\infty} \frac{\del f}{\del x_j}(x) \md \lambda (x_j) \darrow{\text{kompakter Träger}}{=}0 \: \: \: (\text{für} j=2, \dots, k) \\ 
					& \int_{-\infty}^{\infty} \frac{\del f}{\del x_1}(x) \md \lambda (x_1) = f_1 (0, \underbrace{x_2, \dots, x_k}_{x' \in \R^{k-1}})
					\end{align*}
					also
					\begin{equation*}
					\int_{\{x_1 \leq 0\}} \md \omega = \int_{\R^{k-1}} \md \omega = \int_{\R^{k-1}}	f_1 (0,x') \md^{k-1} (x')
					\end{equation*}
					
					Für die Berechnung des Randintegrals bezeichnen wir die Elemente aus $\R{k-1}$ wieder mit $x' = (x'_1, \dots, x'_{k-1})$ und schreiben dementsprechend $\md x'_j, \: j=1, \dots, k-1$. Die Identität $\psi = \id_{\R^k}$ induziert die lokale Parametrisierung
					
					\begin{align*}
					\wt{\varphi} \colon \R^{k-1} &\to  \del \{x_1 \leq 0\} = \{x_1 = 0\} \sbs \R^k \\  x' &\mapsto (0, x')
					\end{align*}
					
					Nach Satz 18 haben wir $\wt{\varphi}^* \md x_j \overset{S.18}{=} \md \tilde{\varphi}^* = \md (x_j \circ \varphi) = \begin{cases*}
					0, \: j=1 \\ \md x_{j-1} \colon j= 2, \dots, k
					\end{cases*} $ \\
					
					Mit Lemma 9 folgt
					\begin{align*}
					\int_{\del \{x_1 \leq 0 \}} \omega = \int_{\R^{k-1}} \wt{\varphi}^* \omega &=\int_ {R^{k-1}} f_1 \abs{\wt{\varphi}(x')} \md x'_1 \wedge \dots \wedge \md x' _{k-1} \\ 
					&= \int _{R^{k-1}} f_1 (0,x') \md \lambda^{k-1}(x')
					\end{align*}
				\end{proof}
			\end{lem}
	
		\begin{theorem}[Glatte Partition der Eins]
			Sei $K \sbs \Rn$ kompakt und $(U_j)_{j=1, \dots, N}$ eine offene Überdeckung von $K$, also $K \sbs \bigcup_{j =1}^{N} U_j$. Dann gibt es eine Überdeckung $(U_j)_{j = 1, \dots, N}$ untergeordnete Partition der Eins $(\alpha_j)_{j = 1, \dots, N}$ mit $\alpha_j \in C^{\infty} (\Rn)$ und $\supp \alpha_j \sbs U_j j = 1, \dots, N$
		\end{theorem}
	
\end{document}