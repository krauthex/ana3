\documentclass[skript.tex]{subfiles}

\begin{document}
	\begin{theorem}[Plancherel]
		Die Foureiertransformation $\F$ lässt sich zu einer linearen beschränkten Abbildung $\wt{\F} \colon L^2(\R^n) \to L^2(\R^n)$ fortsetzen, die \emph{unitär} ist, d.\,h.
		\[
			\skp{\wt{\F}(f)}{\wt{\F}(g)}_{L^2}
			= \skp{f}{g}_{L^2} \quad \forall f,g\in L^2(\R^n).
		\]
	\end{theorem}
	\begin{proof}
		Nach \emph{Satz III.21} liegt $C_c^\infty(\R^n) \sbs \ms{S}(\R^n)$ dicht in $L^2(\R^n)$.\\
		Nach \emph{Lemma 8} ist $\F$ eine lineare und bijektive Selbstabbildung des Schwarzraumes und wegen $\ms{S}(\R^n) \sbs F^1(\R^n)$ erhalten wir die Beschränktheit aus der Plancherel-Identität (\emph{Lemma 12}). Nach \emph{Satz 14} gibt es eine eindeutige lineare Fortsetzung $\wt{\F} \colon L^2(\R^n) \to L^2(\R^n)$ mit $\norm{\wt{\F}(f)}_{L^2} \leq \norm{f}_{L^2}$.\\
		Dass $\wt{\F}$ unitär ist, erhält man mit \emph{Lemma 12} aus
		\begin{align*}
			4\skp{f}{g}_{L^2} &= 4 \int_{\R^n}f(x)\ol{g(x)} \md\lambda^n(x) \\
			&= \norm{f+g}_{L^2}^2 - \norm{f-g}_{L^2}^2 + i\norm{f-ig}_{L^2}^2 - i\norm{f+ig}_{L^2}^2 \\
			&\uarrow{\text{\emph{Lemma 12}}}{=} \dotso = 4\skp{\wt{\F}(f)}{\wt{\F}(g)}_{L^2}.
		\end{align*}
	\end{proof}
	\begin{bem*}
		Solange der Integrand von $\wh{f}$ in $L^1(\R^n)$ liegt, lässt sich $\wt{\F}(f)$ direkt mit der Formel aus \emph{Definition 1} berechnen. In der Regel lässt sich $\wt{\F}$ für $f \in L^2(\R^n)$ nur als Grenzwert einer Folge $\wh{f_k}, \quad(f_k)_{k\in\N} \sbs \ms{S}(\R^n),\quad f_k \xrightarrow[\text{in }L^2]{k\to\infty}f$, darstellen.
	\end{bem*}

	\begin{lem}
		Wir haben $\norm{\wt{\F}(f)}_{L^\infty} \leq (2\pi)^{-\nicefrac{n}{2}} \norm{f}_{L^1}$ für alle $f \in L^1 \cap L^2$.
	\end{lem}
	\begin{proof}
		Sofern $f \in L_c^1(\R^n)$ gilt, folgt $f \in L^2(\R^n)$ und nach \emph{Lemma III.20} konvergiert\\ $\underbrace{f \ast \phi_\epsilon}_{\mathrlap{\in C_c^\infty(\R^n)\sbs\ms{S}(\R^n)}}$ für eine geeignete approximative Identität in $C_c^\infty(\R^n)$ bezüglich der $L^1$-Norm und der $L^2$-Norm gegen $f$. Die Behauptung gilt für $f \ast \phi_\epsilon$ nach \emph{Lemma 2} und für $\epsilon \searrow 0$ erhalten wir das gewünschte mit der Stetigkeit der Norm.\\
		Für allgemeine $f \in L^1(\R^n) \cap L^2(\R^n)$ betrachten wir $f_R=f\chi_{B_R(0)}$.\\ % or \rchi - but change rest of the document, too
		Dann ist $f_R\xrightarrow{R\nearrow\infty}f$ in $L^1$ und $L^2$ (dominierte Konvergenz) und wir erhalten die Behauptung durch Approximation.
	\end{proof}
	Insbesondere gilt für $f \in L^2(\R^n)$\quad($\implies  f\chi_{B_R(0)} \to f$ in $L^2$)
	\[
		\wt{\F}(f)(p) = \lim_{R\nearrow\infty} \frac{1}{(2\pi)^{\nicefrac{n}{2}}} \lint_{B_R(0)} e^{-ipx} f(x) \md\lambda^n(x),
	\]
	wobei der Grenzwert bezüglich der $L^2$-Norm zu verstehen ist und für $f \in F^1(\R^n)$ entfallen kann.
	\begin{lem}
		Für $f,g\in\ms{S}(\R^n)$ gilt $f \ast g$, $fg\in\ms{S}(\R^n)$ und wir haben
		\[
			\wh{f \ast g} = (2\pi)^{\nicefrac{n}{2}} \wh{f}\wh{g} \quad\text{und}\quad \wh{fg}=(2\pi)^{-\nicefrac{n}{2}}\wh{f}\ast\wh{g}.
		\]
		Für $f,g \in L^1(\R^n)$ erhalten wir aus $f \ast g \in L^1(\R^n)$ die erste Abschätzung.
	\end{lem}
	\begin{proof}
		Für $f,g\in\ms{S}(\R^n)$ sind $f \ast g, fg \in \ms{S}(\R^n)$ und wir erhalten $f \ast g \in L^1(\R^n)$ aus der Young-Ungleichung (\emph{Lemma III.17(iv)}).\\
		Nun ist
		\[
			\wh{f \ast g}(p) = \frac{1}{(2\pi)^{\nicefrac{n}{2}}} \lint_{\R^n} e^{-ipx} \lint_{\R^n} f(y)g(x-y) \md\lambda^n(y)\md\lambda^n(x).
		\]
		Nun erhalten wir mit Fubini
		\begin{align*}
			\wh{f \ast g}(p) &= \frac{(2\pi)^{\nicefrac{n}{2}}}{(2\pi)^{\nicefrac{n}{2}}} \lint_{\R^n} e^{-ipy} f(y) \underbrace{\frac{1}{(2\pi)^{\nicefrac{n}{2}}} \lint_{\R^n} e^{-ip(x-y)} \md\lambda^n(x)}_{= \wh{g}(p)}\md\lambda^n(y) \\
			 &= (2\pi)^{\nicefrac{n}{2}} \wh{f}\wh{g}.
		\end{align*}
		Wegen $\ms{S}(\R^n) \sbs L^1(\R^n)$ erhalten wir aus dem soeben Gezeigten für $f,g\in\ms{S}(\R^n)$
		\[
				\wh{f \ast g}(p) = (2\pi)^{\nicefrac{n}{2}} \wh{f}\wh{g} \in \ms{S}(\R^n),
		\]
		denn nach \emph{Lemma 8} haben wir zunächst $\wh{f},\wh{g}\in\ms{S}(\R^n)$ und folglich $\wh{f}\wh{g} \in \ms{S}(\R^n)$.\\
		Weil $\F$ nach \emph{Satz 10} injektiv auf $L^1(\R^n)$ ist, erhalten wir
		\[
			f \ast g = \F^{-1}(\wh{f \ast g}) = (2\pi)^{\nicefrac{n}{2}}\, \F^{-1}(\underbrace{\wh{f}\wh{g}}_{\in \ms{S}}) \darrow{\text{\emph{Corollar 11}}}{\in} \ms{S}(\R^n).
		\]
		Nach \emph{Corollar 11} gilt diese Identität auch bei Ersetzen von $(f,g)$ durch $(\wc{f},\wc{g})$ und so erhalten wir für alle $p \in \R^n$
		\begin{align*}
			(2\pi)^{-\nicefrac{n}{2}}(\wh{f}\ast\wh{g})(p) &= (2\pi)^{-\nicefrac{n}{2}}(\wc{f}\ast\wc{g})(-p) = \F^{-1}\left(\wh{\wc{f}}\wh{\wc{g}}\right)(-p)\\
			&= \F^{-1}(fg)(-p) = \wc{fg}(-p) = \wh{fg}(p).
		\end{align*}
	\end{proof}

	\begin{cor}
		Die Faltung zweier Funktionen $f,g \in L^2(\R^n)$ liegt im Bild von $\F$, welches seinerseits eine Teilmenge von $C_0^0(\R^n)$ ist, und wir haben $\norm{f \ast g}_{L^\infty} \leq \norm{f}_{L^2} \norm{g}_{L^2}$ sowie
		\[
			\uarrow{fg \in L^1}{\wh{fg}} = (2\pi)^{-\nicefrac{n}{2}}\, \wt{\F}(f)\wt{\F}(g).
		\]
	\end{cor}
	\begin{proof}
		%TODO
	\end{proof}
\end{document}
