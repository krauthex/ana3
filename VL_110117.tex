\documentclass[skript.tex]{subfiles}

\begin{document}
\setcounter{chapter}{4}
\setcounter{section}{1}

\textbf{Motivationsfragen:}
\begin{itemize}
	\item Was ist denn oben und was ist unten? (Orientierbarkeit)
	\item Wie sieht es mit Rändern aus, wann sind sie ''glatt'', wann sind sie selbst Mannigfaltigkeiten?
\end{itemize}

\section{Integration auf Mannigfaltigkeiten}
\textbf{Ziel:} Verallgemeinerung des Lebesgue-Integrals auf Mannigfaltigkeiten.

\textbf{Strategie:}
\begin{enumerate}
	\item Nach Definition können wir uns eine Mannigfaltigkeit $M$ lokal mit einer Immersion $\varphi \colon U\sbs\Rn \lra \Rn$ parametrisieren -- also betrachten wir zunächst Stücke von Mannigfaltigkeiten.
	
	\item Eine ''beliebige'' Mannigfaltigkeit zerlegen wir in endlich viele überlappende Teilstücke mit lokalen Parametrisierungen (wir setzen also voraus, dass $M$ einen endlichen Atlas besitzt). Der Wert des Integrals ist unabhängig vom gewählten Atlas. Wichtigstes Hilfsmittel: Partition der Eins.
\end{enumerate}

Zunächst der lineare Fall. Für eine lineare Abbildung $T \colon \R^m \lra \Rn, n\geq m,$ und eine messbare Menge $U\sbs \R^m$ möchten wir den ''m-dimensionalen Flächeninhalt'' von $T(U)$ angeben. Wir erwarten, dass der n-dimensionale ''Flächeninhalt'' von $T(U)$ im Fall $m<n$ Null beträgt\footnote{Siehe \textit{Aufgabe 18}: $\mbb{S}^{n-1}$ ist eine $\lambda^n$-Nullmenge.}.

\begin{lem}
	Sei $T\in\R^{n\times m}$ (charakterisiert lineare Abbildung $\R^m \lra \Rn$) mit Rang $m, n\geq m$. Dann gibt es $Q \in \R^{n\times m}$  und $S \in \R^{m\times m}$ mit $T=QS$, wobei $Q$ eine \textup{Isometrie} ist, das heißt $\abs{Qv}_\Rn = \abs{v}_{\R^m} \forall v \in \R^m$ und $\abs{\det S} = \sqrt{\det T^\tp T }$.
\end{lem}

\begin{tikzpicture}
	\node at (-2,0) (nullpunkt) {};
	
	\draw [<->, >=latex] (0,3) -- (0,0) -- (3,0); 
	\squigglyset{0.5}{.6}{1.2}{.7};
	\node at (1.5, 1.3) (U1)  [green] {$U$};
	\node at (3, 2.5) (Raum1) {$\R^m$};
	
	\draw [->, >=latex] (3.2, 1.5) -- node [above] {$T$} (4.2, 1.5);
	
	\draw [<->, >=latex, thin] (5,3) -- (5,0) -- (8,0);
	\node at (8, 2.5) (Raum2) {$\Rn$};
	\draw [thick] (4.5, 0.5) -- (7,2) -- (7,3) -- (5.05,1.81);
	\draw [thick] (4.95, 1.78) -- (4.5, 1.5) -- (4.5,0.5);
	\squigglyset{5.8}{1.7}{0.4}{.3};
	\draw (5.7, 1.5) -- (7, 1) node [right] {$T(\R^m)$};
	\draw (6,1.9) -- (5.5, 3) node [above, green] {$T(U)$};
\end{tikzpicture}

\begin{proof}
	Mit $e_1, \dots, e_m$ bezeichnen wir die Standard-Einheitsbasis des $\R^m$. Weiter wählen wir eine Orthonormalbasis $\cbr{f_1, \dots, f_n}$ des $\Rn$ mit $T(\R^m) = \spann \cbr{f_1, \dots, f_m}$.
	
	Nun Können wir $Q \in \R^{n\times m}$ durch $Qe_j = f_j, j=1,\dots,m$ eindeutig definieren und für 
	\[
		v = \sum_{j=1}^{m} v_j e_j,\quad w = \sum_{k =1}^m w_k e_k
	\]
	erhalten wir
	\[
		\skp{Qv}{Qw}_{\Rn} \stackrel{\footnotemark}{=} \sum_{j,k} v_j w_k \delta_{jk} = \sum_{j=1}^m v_j w_j = \skp{v}{w}_{\R^m},
	\]
	\footnotetext{$\skp{f_j}{f_k}_{\R^m} = \delta_{jk}$}
	also $\abs{Qv}_{\Rn} = \abs{v}_{\R^m} \forall\ v\in \R^m$, und
	\[
		\skp{Q^\tp Qv}{w}_{\R^m} = w^\tp Q^\tp Qv = (Qw)^\tp Qv = \skp{Qv}{Qw}_{\Rn} = \skp{v}{w}_{\R^m},
	\]
	woraus $Q^\tp Q = \id_{R^m}$ ersichtlich ist. Nach Konstruktion ist $Q$ auf $T(\R^m)$ invertierbar, somit ist $S \ceq Q^{-1} T \in \R^{m\times m}$ wohldefiniert, und $\rang S = m$. Wir haben $QS = T$ und 
	\begin{align*}
		\det\pr{T^\tp T} = \det\pr{(QS)^\tp QS} &= \det\big( S^\tp \underbrace{Q^\tp Q}_{\mathclap{\id_{\R^m} }} S\big)\\
		&= \det\pr{S^\tp S}\\ 
		&= \det(S) \det\pr{S^\tp}\\
		&= \abs{\det S}^2.
	\end{align*}
\end{proof}

Wir möchten einen ''Flächeninhalt'' definieren, der invariant unter Isometrien (Translation, Rotation, Spiegelung) ist. Insofern sollte der Flächeninhalt von $T(U)$ mit dem von \\$Q^{-1} T(U) = S(U)$ übereinstimmen. Wegen $S(U) \sbs \R^m$ können wir das m-dimensionale Lebesgue-Maß verwenden und erhalten:
\[
	\lambda^m (S(U)) = \lint_{S(U)} \mathds{1} \md \lambda^m \darrow{\textup{I.69c}}{=} \abs{\det S} \lint_U \mathds{1} \md \lambda^m = \sqrt{\det \pr{T^\tp T}} \lambda^m(U).
\]


\begin{defin}[Integral auf lokaler Parametrisierung]
	Seinen $m,n \in \N, m\leq n, U \sbs \Rn$ offen, $\varphi \in \C^1(U,\Rn)$ eine Immersion, die $U$ homöomorph auf $\image \varphi$ abbildet. Dann definieren wir den mehrdimensionalen Flächeninhalt von $\image \varphi$ durch
	\[
		\vol^m(\image \varphi) = \int_U \sqrt{\det \pr{(D\varphi)^\tp (D\varphi)}} \md \lambda^m ,
	\]
	wobei $\det \pr{(D\varphi)^\tp (D\varphi)}$ mit \textit{Gram-Determinante} bezeichnet wird.
	
	Eine Funktion $f\colon \image \varphi \lra \R$ heißt integrierbar, falls
	\[
		(f \circ \varphi) \sqrt{\det \pr{(D\varphi)^\tp (D\varphi)}}
	\]
	auf $U$ integrierbar ist.
	
	Das m-dimensionale Flächenintegral auf $\image \varphi$ ist durch
	\[
		\lint_{\image \varphi} f \md A^m = \lint_U (f\circ \varphi) \sqrt{\det \pr{(D\varphi)^\tp (D\varphi)}} \md \lambda^m
	\]
	gegeben. Entsprechend sind die Räume $L^p (\image \varphi)$ erklärt.
	
	Im Fall $n=m$ ergibt sich mit $\varphi = \id$:
	\[
		\int_U f \md A^n = \int_U f \md \lambda^n.
	\]
\end{defin}

\begin{lem}[Wohldefiniertheit des Flächeninhalts]
	Seien $n,m \in \N, m\leq n, U_1, U_2 \sbs \R^m$ offen und $\varphi_1 \in C^1(U_1, \Rn), \varphi_2 \in C^1(U_2, \Rn)$ Immersionen, die $U_1$ beziehungsweise $U_2$ homöomorph auf eine Menge $W\sbs \Rn$ abbilden. Sei weiterhin $f\colon W \lra \R$ messbar. Dann ist $(f\circ \varphi_1) \sqrt{\det \pr{(D\varphi_1)^\tp (D\varphi_1)}}$ genau dann integrierbar, wenn $(f \circ \varphi_2) \sqrt{\det \pr{(D\varphi_2)^\tp (D\varphi_2)}}$ integriebar ist und wir haben
	\[
		\lint_{U_1} (f \circ \varphi_1) \sqrt{\det \pr{(D\varphi_1)^\tp (D\varphi_1)}} \md \lambda^m = \lint_{U_2} (f\circ \varphi_2) \sqrt{\det \pr{(D\varphi_2)^\tp (D\varphi_2)}} \md \lambda^m.
	\]
\end{lem}

\begin{proof}
	Wir setzen $\psi = \varphi_1^{-1} \circ \varphi_2 \colon U_2 \lra U_1$. Wenn wir zeigen, dass $\psi$ ein Diffeomorphismus ist, so folgt die Behauptung mit dem \textit{Transformationssatz (I.70)} wegen
	\[
		\lint_{U_1} (f\circ \varphi_1) \sqrt{\det \pr{(D\varphi_1)^\tp (D\varphi_1)}} \md \lambda^m = \lint_{U_2} (f\circ \underbrace{\varphi_1 \circ \psi)}_{\varphi_2} \sqrt{\det \pr{(D\varphi_1 \circ \psi)^\tp (D\varphi_1 \circ \psi)}} \abs{\det D\psi} \md \lambda^m,
	\]
	woraus wir mit $D\varphi_2 = D(\varphi_1 \circ \psi) = (D\varphi_1 \circ \psi) D\psi$ und
	\begin{align*}
		\det \pr{(D\varphi_2)^\tp (D\varphi_2)} &= \det \big( (D\psi)^\tp \underbrace{(D\varphi_1 \circ \psi)^\tp (D\varphi_1 \circ \psi)}_{\in \R^{m\times m}} (D\psi) \big)\\
		&= \det (D\psi) \det \pr{(D\varphi_1 \circ \psi)^\tp (D\varphi_1 \circ \psi)} \det (D\psi)\\
		&= \abs{\det D\psi}^2 \det \pr{(D\varphi_1 \circ \psi)^\tp (D\varphi_1 \circ \psi)}
	\end{align*}
	die geforderte Identität erhalten.
	
	Es bleibt zu zeigen, dass $\psi$ ein Diffeomorphismus ist. Als Verkettung zweier Homöomorphismen ist $\psi$ wieder ein Homöomorphismus. Insofern genügt der Nachweis, dass $\psi, \psi^{-1} \in C^1$ sind. Aus Symmetriegründen können wir uns auf $\psi \in C^1$ beschränken. 
	
	Zur Charakterisierung von $\varphi_1^{-1}$ fixieren wir ein beliebiges $u_2 \in U_2$ und setzen $x=\varphi_2(u_2),\\u_1=\varphi_1^{-1}(x) = \psi(u_2)$. Die Spalten von $D\varphi_1(u_1)$ spannen den Tangentialraum $T_x W$ an $W$ im Punkt $x$ auf. Mit $P$ bezeichnen wir die Projektion $P\colon \Rn \lra T_x W \cong \R^m$. Nach der Kettenregel haben wir für $P \circ \varphi_1 \colon U_1 \lra T_x W$
	\[
		D(P\circ \varphi_1)(u_1) = DP\big( \underbrace{\varphi_1(u_1)}_{x} \big) D\varphi_1(u_1) = P\pr{D\varphi_1(u_1)} = D\varphi_1(u_1).
	\]
	Insbesondere ist $\rang D(P\circ\varphi_1)(u_1) = \rang D\varphi_1(u_1) = m$.
	
	Nach dem Umkehrsatz ist $P\circ \varphi_1$ in einer Umgebung von $u_1$ invertierbar. Folglich gibt es Umgebungen $\wt{U}_1 \sbs \R^m, \wt{W} \sbs T_x W$ mit $u_1 \in \wt{U}_1$ und $g \in C^1(\wt{W}, \wt{U}_1)$ mit $g \circ P \circ \varphi_1 = \id_{\wt{U}_1}$. Wir setzen $\wt{U}_2 = \varphi_2^{-1}(\wt{W})$ und erhalten $\psi = \varphi_1^{-1} \circ \varphi_2 = g \circ P \circ \varphi_2$ auf $\wt{U}_2$. Insbesondere ist $\psi|_{\wt{U}_2} \in C^1(\wt{U}_2, \R^m)$. Da $u_2$ beliebig war, folgt $\psi \in C^1(U_2, \R^m)$.
	
	\begin{tikzpicture}
		\node at (-1,0) (nullpunkt) {};
		\manifold{2.5}{3}{1.5};
		\squigglyset{0}{0}{1.5}{1};
		\squigglyset{10}{-0.5}{1}{1.3};
		
		% points
		\filldraw [red] (1.5,1) circle[radius=.05] node[above]{$u_1$};
		\filldraw [red] (10.5,1) circle[radius=.05] node[above]{$u_2$};
		\filldraw [red] (6.3,4.3) circle[radius=.05] node[above]{$x$};
		
		% spaces
		\node at (1, 2.5) (U1) {$U_1$};
		\node at (8.5,5) (W) {$W$};
		\node at (11, 2.7) (U2) {$U_2$};
		
		% maps
		\draw[->, >=latex, thick] (4.8,3.2) to[bend right] node[below]{$\varphi_1^{-1}$} (2.3,1.8);
		\draw[->, >=latex, thick] (10,1.7) to[bend right] node[above]{$\varphi_2$} (8,3);
		\draw[->, >=latex, thick, green] (9.5,.5) to[bend right] node[above, green]{$\psi$} (2.6, 0.7);
		
		% tangent space
		\draw [dashed, rotate around={-17:(6.3,4.3)}] (5.5,3) rectangle (7, 5.5);
		\node at (5, 5.5) (TxW) {$x+T_x W$};
	\end{tikzpicture}
\end{proof}
\end{document}